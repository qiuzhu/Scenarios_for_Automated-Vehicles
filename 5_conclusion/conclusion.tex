\section{结论与展望}
\label{conclusion}
%In this paper, the authors analyzed the practicability of a scenario-based approach for the design of vehicle guidance systems following the development process of the ISO 26262 standard.
本文分析了基于场景的方法在依据ISO 26262标准开发自动驾驶系统过程中的可行性。
%analyzed the development process of the ISO~26262 standard regarding the practicability of a scenario-based development for vehicle guidance systems.
%For this purpose, the process steps in which scenarios may be used to generate the work products of the respective process step have been identified.
%Furthermore, requirements regarding the representation of scenarios have been defined and contradictions regarding the requirements resulting from different process steps have been shown.
作者分析了可以使用场景来生成工作输出产品的各个工作阶段,并明确了不同阶段对场景描述的需求,阐述了场景描述需求在细节程度上存在的差异。
%On this basis, the authors suggested three levels of abstraction for scenarios in order to fulfill all requirements defined above.
在此基础上,作者定义了场景的三个抽象级别,以满足上文阐述的场景需求。
%Furthermore, a definition for each introduced level of abstraction has been given and it has been shown, how the levels of abstraction for scenarios can be used to generate work products for different process steps defined in the ISO~26262 standard.
此外,作者给出了每个抽象级别的定义,并说明了如何使用场景的抽象级别来生成ISO~26262标准中定义的不同阶段的工作产品。
%In the future, new methods and tools are needed to generate functional scenarios and to convert these functional scenarios to concrete scenarios along the development process of the ISO~26262 standard. 
未来,需要新的方法和工具来生成功能场景,并将这些功能场景转换为ISO~26262标准开发过程中的具体场景。
%In addition to this contribution, there is a companion contribution submitted to the 2018 IEEE Intelligent Vehicles Symposium with an knowledge based approach for creating functional scenarios with a large variety.
除了这一贡献之外,还提交了2018年IEEE智能车辆研讨会的配套文稿,其中提供了基于知识的方法,用于创建各种各样的功能场景。
%Therefore, existing data formats for scenarios can be integrated into the suggested levels of abstraction.
因此,场景的现有数据格式可以集成到建议的抽象级别中。
%Afterwards, new methods and tools for scenario specification and scenario concretization can be developed with respect to a test concept for automated vehicles.
之后,可以针对自动驾驶车辆的测试概念,开发用于场景规范和场景具体化的新方法和工具。