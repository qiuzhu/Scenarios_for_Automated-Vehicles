\section{前言}
目前市场上已有SAE等级1和2 \cite {SAE2016}的驾驶员辅助系统和自动化系统
(奥迪交通堵塞试点\cite{noauthor_techday_nodate}或Waymo自动驾驶汽车\cite{hawkins_waymo_2017})宣布遵循3级(有条件自动化)和4(高度自动化)。
%A challenge for the introduction of higher levels of automation is to assure that these vehicle systems behave in a safe way.
引入更高级自动化的挑战是确保这些车辆系统以安全的方式运行。
%For driver assistance systems, this proof is furnished by driving many test kilometers on test grounds and public roads.
对于驾驶辅助系统,安全证明是通过在测试场地和公共道路上驾驶里程来提供的。
%However, for higher levels of automation a distance-based validation is not an economically acceptable solution \cite{wachenfeld_release_2016}. 
但对于更高级别的自动化,基于测试里程的验证的解决方案在经济成本上不可接受。\cite{wachenfeld_release_2016}。
%As an alternative to the distance-based validation we introduce a scenario-based approach.
作为基于距离的验证的替代方案,我们引入了基于场景的方法。
%The key idea is to purposefully vary and validate the operating scenarios of the automated vehicle.
其中的关键是有目的的改变和验证自动驾驶汽车的运行场景。
%Therefore, the systematic derivation of scenarios and further assumptions have to be documented along the development process to ensure a traceable scenario generation.
因此,必须在开发过程中系统的记录与推演场景,以确保生成可追踪的场景。
%Therefore, scenarios have to be systematically documented and derived along the development process to ensure a traceable scenario generation.
%A challenge for the introduction of higher levels of automation is the development, the verification, and the validation of safety concepts for such vehicle guidance systems.

%The ISO~26262 standard is a guideline for the development of safety-critical electric/electronic vehicle systems and thus provides a framework for the development of vehicle guidance systems under the aspect of functional safety.
%According to the ISO~26262 standard, scenarios can be utilized to support the development process.
ISO~26262标准是开发安全关键电气/电子车辆系统的指南,因此为功能安全方面的车辆引导系统的开发提供了框架。根据ISO~26262标准,可以利用场景来支持开发过程。
%For instance, scenarios can help to derive requirements, to develop the necessary hardware and software components, and to prove the safety of these components in the test process. 
%When creating test cases, scenarios are necessary for generating consistent input data for the test object in any case.
例如,场景有助于推导需求,开发必要的硬件和软件组件,并在测试过程中证明这些组件的安全性。
%Nevertheless, these different applications of scenarios result in distinct requirements for scenario representation in each development phase of the ISO~26262 standard. 
然而,这些场景的不同应用导致对ISO~26262标准的每个开发阶段中的场景表示的不同要求。

%This contribution proposes three abstraction levels for scenarios along a V-model-based development process.
%In this way, scenarios can be identified on a high level of abstraction in the concept phase and be detailed and concretized along the development process. 
本文贡献为基于V模型的开发过程中的场景提出了三个抽象级别。
通过这种方式,可以在概念阶段的高级抽象中识别场景,并在开发过程中进行详细和具体化。
%This allows a structured approach, starting from the item definition according to the ISO~26262 standard, followed by the hazard analysis and risk assessment (HARA), and ending up with the necessary test cases for safety verification and validation.
这允许采用结构化方法,从根据ISO~26262标准的项目定义开始,然后进行危害分析和风险评估(HARA),最后得到必要的安全验证和验证测试用例。
%Thus, the authors suggest an extended definition of the term `scenario' based on the definition of Ulbrich~et~al.~\cite{ulbrich_definition_2015} and introduce the abstraction levels of functional, logical, and concrete scenarios. 
因此,作者基于Ulbrich等人\cite{ulbrich_definition_2015}的定义提出了对“场景”这一术语的扩展定义,并介绍了功能,逻辑和具体场景的抽象级别。
%A German version of this paper has been published at a workshop on driver assistance systems \cite{SzenarienProzess2017}.

%The paper is structured as follows:
本文组织结构如下:
%Section~\ref{related_work} gives a short motivation based on selected related work regarding scenarios in the development process for automated driving functions, utilized levels of abstraction for scenarios, and existing definitions of the term scenario.
第\ref {related_work}节基于选定的相关工作提供了一个简短的动机,这些工作涉及自动驾驶功能的开发过程中的场景,场景的抽象使用级别以及术语场景的现有定义。
%Section~\ref{process} derives and analyzes requirements for the representation and usage of scenarios in the development process of the ISO~26262 standard.
第~\ref{process}节推导并分析ISO~26262标准开发过程中场景表示和使用的要求。
%Afterwards, section~\ref{terminologie} defines three layers of abstraction for scenarios and shows how these scenario representations can be converted into each other along the development process.
之后,第\ref{terminologie}节部分为场景定义了三层抽象,并展示了这些场景表示如何在开发过程中相互转换。
%Finally, section~\ref{conclusion} gives a short conclusion.
最后,第~\ref{conclusion}节作出一份总结。