\section{相关工作}
\label{related_work}
%
%Ulbrich~et~al.~\cite{ulbrich_definition_2015} analyze the term \emph{scenario} across multiple disciplines and propose a consistent definition for the domain of automated vehicles.
Ulbrich等人.~\cite{ulbrich_definition_2015}分析跨多个学科的术语\emph {场景},并提出自动化车辆领域的一致定义。
%In this paper, the authors use the term scenario referring to the definition of Ulbrich~et~al.~\cite{ulbrich_definition_2015}.
本文中作者引用的场景定义源自\cite{ulbrich_definition_2015}中的场景定义。

%Go~and~Carroll~\cite{go_blind_2004} point out that scenarios have a different use across various disciplines, but the elements utilized to describe a scenario are similar in all cases.
%Thereby, scenarios can be described in several levels of detail and different forms of notation.
Go~和~Carroll~\cite{go_blind_2004}指出场景在不同学科中有不同的用途,但用于描述场景的元素在所有情况下都是相似的。
因此,可以在几个细节层次和不同形式的符号中描述场景。
%Scenarios may be expressed in formal, semi-formal, or informal notation \cite{go_blind_2004}.
场景可以用正式,半正式或非正式表示法表达\cite{go_blind_2004}。
%This distinction hints at multiple levels of abstraction of scenarios along the development process for automated vehicles.
这种区别暗示了自动车辆开发过程中多个场景的抽象层级。

%Bergenhem~et~al.~\cite{bergenhem_how_2015} point out that complete requirements for vehicle guidance systems\footnote{To the authors' opinion, it is impossible to generate a complete set of requirements for higher levels of automation.} can only be achieved by a consistent, traceable, and verifiable process of requirements engineering in accordance with the V-model. %\cite{VDI2206}. 
%Bergenhem等人.~\cite {bergenhem_how_2015}指出车道引导系统的完整要求\footnote{作者认为不可能为更高级别的自动化生成一整套要求。}只能是通过可追溯和可验证的需求,符合V模型开发的工程流程实现。
%Several publications suggest approaches which utilize scenarios to generate work products along the development process for automated vehicles.
一些论文提出了利用场景来生成自动驾驶车辆开发过程中的工作产品的方法。
%Bagschik~et~al.~\cite{bagschik_identification_2016} develop a procedure for the generation of potentially hazardous scenarios within the process step of a hazard analysis and risk assessment, as suggested by the ISO~26262 standard.
Bagschik等人.~\cite{bagschik_identification_2016}在危险分析和风险评估的过程步骤中开发了符合ISO~26262标准的产生潜在危险情景的程序。
%This procedure utilizes an abstract description of the traffic participants and the scenery in natural language.
该过程利用了交通参与者的抽象描述和自然语言的情景。
%All possible combinations of scenario elements are analyzed incorporating descriptions of functional failures in a limited use case of an SAE Level 4 \cite{SAE2016} vehicle guidance system within the scope of the project \textit{Unmanned Protective Vehicle for Highway Hard Shoulder Road Works} (aFAS\footnote{This abbreviation is derived from the German project name.}) \cite{stolte_towards_2015}. 
文章中分析了所有可能的场景元素组合,纳入自动驾驶车辆项目范围内的SAE L4~\cite{SAE2016}有限用例中功能失效的描述。
%Schuldt~et~al.~\cite{schuldt2011} motivate a scenario-based test process and present a systematic test case generation by use of a 4-layer-model.

Schuldt等人~\cite {schuldt2011}开发基于场景的测试过程,通过使用4层模型生成系统测试用例。
%Bach~et~al.~\cite{bach_model_2016} propose a model-based scenario representation with spatial and temporal relations as a general scenario notation along the development process of the ISO~26262 standard.
Bach~et~al.~\cite{bach_model_2016}提出了一种基于模型的场景描述方式,其具有空间和时间关系,作为ISO~26262标准的开发过程中的一般场景描述。
%This scenario representation is implemented prototypically for scenarios of an ACC-system on motorways and the results are presented.
该场景描述是针对高速公路上的ACC系统的场景原型实现的,并且呈现了结果。

%The mentioned publications utilize scenarios with different levels of abstraction for the functional and safety development of vehicle guidance systems.
所提到的论文利用具有不同抽象层级的场景来实现车辆引导系统的功能和安全开发。
%The term `scenario' has not been defined uniformly, which makes it difficult to achieve a consistent understanding regarding the role of scenarios in the development process.
术语“场景”尚未统一定义,这使得难以对场景在开发过程中的作用达成一致的理解。
%For this reason, the authors will derive and analyze requirements on scenarios in the following part.
因此,作者将在以下部分中推导并分析有关场景的需求。