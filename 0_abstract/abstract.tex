%Einleitender Satz?
%The latest version of the ISO~26262 standard from 2016 represents the state of the art for a safety-guided development of safety-critical electric/electronic vehicle systems.
%These vehicle systems include advanced driver assistance systems and vehicle guidance systems.
%The development process proposed in the ISO~26262 standard is based upon multiple V-models, and defines activities and work products for each process step.
%In many of these process steps, scenario based approaches can be applied to achieve the defined work products for the development of automated driving functions.
%To accomplish the work products of different process steps, scenarios have to focus on various aspects like a human understandable notation or a description via state variables.
%This leads to contradictory requirements regarding the level of detail and way of notation for the representation of scenarios.
%In this paper, the authors discuss requirements for the representation of scenarios in different process steps defined by the ISO~26262 standard, propose a consistent terminology based on prior publications for the identified levels of abstraction, and demonstrate how scenarios can be systematically evolved along the phases of the development process outlined in the ISO~26262 standard.

2016年更新的ISO~26262标准,代表了车辆安全的关键电气/电子系统安全指导开发的最新技术,可以应用于高级驾驶辅助系统(ADAS)和自动驾驶系统的开发和验证。标准规定了基于V型开发模式的各个阶段所要求的工作内容和输出产品。在V型开发模式的各个阶段,均可应用基于场景的方法,来获得相应的工作输出产品。为了完成各个阶段的工作产品,场景必须关注各种方面,如人类可理解的符号或通过状态变量的描述。
在应用基于场景的方法时,不同开发阶段对场景细节程度和场景描述方式的需求存在矛盾。本文作者讨论了ISO 26262标准中不同阶段对场景描述的要求,提出了满足一致性的场景描述方法,并演示了如何系统建立满足不同阶段需求的场景。
